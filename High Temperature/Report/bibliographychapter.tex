%!TEX root = Report.tex


\chapter{Managing references}\label{c:references}

\section{Citations}\label{s:citations}
Any citation made in the text should refer to an entry in the bibliography which is listed at the end of the report. To cite a reference the \verb+\cite{.}+ command is used. 
A BibTeX file centralizes all informations of the references used in the report.\\

\subsection{Cite Key}
Each BibTeX entry contains a unique \textit{Cite Key}. One way you can form these keys is to use the \textit{first author + year + short title}. These keys are then used to actually link a reference.\\

Taking an article from \textit{Howie Choset} about \textit{Coverage for Robotics} written in 2001 this key would be \textit{choset2001coverage}. To cite this reference, type \verb+cite{choset2001coverage}+. In the text this citation looks like this: \cite{choset2001coverage}. \\

\subsection{Bibliography style}
To include the bibliography, you first define it's style with \verb+\bibliographystyle{.}+. There are several styles to choose from. They define for example how the entries in the bibliography are ordered. This could be in the order they were cited (e.g.~\textit{ieeetr}) or in alphabetical order (e.g.~\textit{plain}). The latter is the default style and will be used when no style is defined. With \verb+\bibliography{.}+ the bibliography is created. \\


\subsection{Additional hints}
Use the command \verb+\nocite{.}+ if you want to list references in the bibliography which are not cited in the text. Placing an asterisk (*) in the \verb+\nocite{*}+ command puts all references which are located in your BibTeX file into the bibliography.

To avoid a listing of successive references such as \cite{bishop2006pattern}  \cite{choset2001coverage}  \cite{gassert20062dof} the \textit{cite} package

 (\verb+\usepackage{cite}+) can be used to combine those lists in the form of \cite{bishop2006pattern,choset2001coverage,gassert20062dof} (\verb+\cite{ . , . , .}+). 
 
 
 
 
 