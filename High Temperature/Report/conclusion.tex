%!TEX root = Report.tex
\chapter{Fehlerquellen und Schlussfolgerungen}\label{sec:conclusion}

Diese Laborübung hat uns gezeigt wie gross die Temperaturabweichung eines Thermoelementes sein kann, wenn höhere Umgebungstemperaturen vorhanden sind und so die Wärmestrahlung an Gewicht zunimmt. Man konnte trotzdem durch relativ wenige Messungen und eine einfache Wärmebilanz mit bekannten Beziehungen aus der Thermodynamik auf eine sehr plausible Approximation für die reale Gastemperatur kommen.\\

Ein klarer Fehler ist allerdings aufgetaucht. Die reale Gastemperatur $T_{gas}$ sollte mit zunehmendem Volumenstrom annährend konstant bleiben. Im Gegensatz dazu ist ein leichter Anstieg zu beobachten. Dieser lässt sich vermutlich auf Modellierungsfehler und vereinfachte Annahmen in den thermodynamischen Beziehungen zurückführen.
