%!TEX root = Report.tex
% Acknowledgment -----------------------------
%\chapter*{Acknowledgments}

%Acknowledge people who helped you and contributed to this project here \dots

 %\cleardoublepage

%---------------------------------------------------------------------------
% Table of contents

 \setcounter{tocdepth}{2}
 \tableofcontents

 %\cleardoublepage

%---------------------------------------------------------------------------
% Lists

 \listoffigures % Creates list of all figures used in document
 \cleardoublepage

 % \listoftables  % Creates list of all tables inserted
 %\cleardoublepage

%---------------------------------------------------------------------------
% Abstract

\chapter*{Abstract}
 \addcontentsline{toc}{chapter}{Abstract}

This experiment helps us to investigate and visualize different flow characteristics like density and temperature through analysis of the refraction index of a transparent liquid. In this experiment we made use of a camera and a noise-background. After having calibrated the measurements with the wedge experiment (measurement of the pixel deviation in the camera with known angle deviation of a certain probe lense) we measured the deviation of the noise background through a heated paraffin-filled probe. Through the mathematical relations between pixel deviation and light refraction were able to obtain very plausible temperature profiles (closely related to the refraction index) at two different times of the heated convection cell filled with paraffin. 
%---------------------------------------------------------------------------
% Symbols

%\chapter*{Symbols}\label{chap:symbole}
% \addcontentsline{toc}{chapter}{Symbols}
%
%\section*{Symbols}
%\begin{tabbing}
% \hspace*{3cm} \= \kill
%  $\phi, \theta, \psi$	\> roll, pitch and yaw angle \\[0.5ex] 					
%  $b$				\> gyroscope bias \\[0.5ex]										
%  $\Omega_m$		\> 3-axis gyroscope measurement \\[0.5ex]   		
% \end{tabbing}
%
%\section*{Indices}
%\begin{tabbing}
% \hspace*{1.6cm}  \= \kill
% $x$ \> x axis \\[0.5ex]
% $y$ \> y axis \\[0.5ex]
% \end{tabbing}
%
%\section*{Acronyms and Abbreviations}
%\begin{tabbing}
% \hspace*{1.6cm}  \= \kill
% ETH \> Eidgenoessische Technische Hochschule \\[0.5ex]
% EKF \> Extended Kalman Filter \\[0.5ex]
% IMU \> Inertial Measurement Unit \\[0.5ex]
% UAV \> Unmanned Aerial Vehicle \\[0.5ex]
% UKF \> Unscented Kalman Filter \\[0.5ex]
%\end{tabbing}
%
% \cleardoublepage

%---------------------------------------------------------------------------
