%!TEX root = Report.tex
\chapter{Summary And Outlook}\label{sec:conclusion}



\section{Recognized Shortcomings}

One of the shortcomings that we could observe in this experiment is the size of the probe used. Altough the wedge experiment for the calibration delivered very good results, the temperature fields could have delivered better results, specially on the edges, if the probe (and thus the area of analysis) would have been bigger. Also, the fact that the temperature field was only calculated for two different times, the results have little significance in terms of time. The camera having itself a curved lense, could have a certain effect on the results depending on how close the lens is to the probe. Ideally, the best results would be delivered by a camera with a high zoom, that is still far away from the probe. This would, of course require a higher quality lens and sensor to maintain the quality of the picture.


\section{Potential Improvements}

The value of the results could be much higher if, for example, there was a bigger batch of analysed pictures at known times. In this case, more conclusions could be made about the properties of the heat distribution in paraffin.


\section{Outlook And General Conclusion}

The method of the Schlieren Analysis taught us how so plausible and reliable results can be taken from such a simple setup (a camera and a background with noise). Of course there is the disadvantage of measuring fluids with high opacity. But this technique allow us, even to perform quality measures (with the help of MATLAB) only with equipment found home.