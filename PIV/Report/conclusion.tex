%!TEX root = Report.tex
\chapter{Conclusion}\label{sec:conclusion}






\section{Suitability of PIV for the measured Flow Field}
Since our flow field is quasi two dimensional and only few velocity compenents normal to the measured plane occur,  the PIV is very well suited. In addition, the Reynolds Number of the fluid is very low and therfore a laminar flow is formed. This is advantageous because in a turbulent flow information is lost. The fluctuations of the velocity that typically occur in turbulent flows can not be registered, as the flow field is divided in small panels and averaged in these panels.\\

One of the main issues in PIV is the framerate of the camera. Only with the development of CCD high speed cameras, the required frame rates are reached. As we have a very slow flow (around $0.5\ mm/s$), the framerate of the camera is hence not an issue and no expensive highspeed camera is required. \\

Due to the optics of the camera, a slight distortion at the border of the image arises. As we have exactly at the same areas the highest displacments/velocities, we need to consider this fact in our interpretation.

\section{Limitations And Possible Improvements}

As it was mentioned before, the change of quality in the pictures wasn't traceable enough with the used shutter aperture and exposure times used. Altough the good plausability of our results, a higher difference in those mentioned parameters would help us to better understand their effects on the final results. \\

The effect of peak loking could be overcome with a filtering of the images taken. An induced blurriness (for example with averaging of the sorroundings of a particle) in the images would make the particles bigger and their movement better traceable for the case of small displacements. This way, the peak locking could decrease and the histogram of displacement would look much more homogeneously distributed.