%!TEX root = Report.tex
% Acknowledgment -----------------------------
%\chapter*{Acknowledgments}

%Acknowledge people who helped you and contributed to this project here \dots

 %\cleardoublepage

%---------------------------------------------------------------------------
% Table of contents

 \setcounter{tocdepth}{2}
 \tableofcontents

 %\cleardoublepage

%---------------------------------------------------------------------------
% Lists

 \listoffigures % Creates list of all figures used in document
 \cleardoublepage

 \listoftables  % Creates list of all tables inserted
 %\cleardoublepage

%---------------------------------------------------------------------------
% Abstract

\chapter*{Abstract}
 \addcontentsline{toc}{chapter}{Abstract}

In this experiment the flow of a one-stage turbomachine is examined. As a measuring tool a five-pinhole-probe is introduced to analyze the pressure and velocity of the flow in different positions behind the stator. All data is stored in LabView, the program that controls the entire measurement process. Finally the information obtained is post processed and visualized in Matlab.\\\\
The flow after the stator is highly unsteady.  Even though the measured steady data is expressive. By averaging over the mass flow some unsteady influences were removed. The effect of the boundary layer around the blade was visible on most of the plots and also the radial pressure difference due to the rotating fluid.\\\\
The performed experiment shows clearly that with this setup it is possible to improve the efficiency of blades by analyzing the pressure and velocity distribution.


%---------------------------------------------------------------------------
% Symbols

%\chapter*{Symbols}\label{chap:symbole}
% \addcontentsline{toc}{chapter}{Symbols}
%
%\section*{Symbols}
%\begin{tabbing}
% \hspace*{3cm} \= \kill
%  $\phi, \theta, \psi$	\> roll, pitch and yaw angle \\[0.5ex] 					
%  $b$				\> gyroscope bias \\[0.5ex]										
%  $\Omega_m$		\> 3-axis gyroscope measurement \\[0.5ex]   		
% \end{tabbing}
%
%\section*{Indices}
%\begin{tabbing}
% \hspace*{1.6cm}  \= \kill
% $x$ \> x axis \\[0.5ex]
% $y$ \> y axis \\[0.5ex]
% \end{tabbing}
%
%\section*{Acronyms and Abbreviations}
%\begin{tabbing}
% \hspace*{1.6cm}  \= \kill
% ETH \> Eidgenoessische Technische Hochschule \\[0.5ex]
% EKF \> Extended Kalman Filter \\[0.5ex]
% IMU \> Inertial Measurement Unit \\[0.5ex]
% UAV \> Unmanned Aerial Vehicle \\[0.5ex]
% UKF \> Unscented Kalman Filter \\[0.5ex]
%\end{tabbing}
%
% \cleardoublepage

%---------------------------------------------------------------------------
